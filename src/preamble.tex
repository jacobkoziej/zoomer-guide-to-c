% SPDX-License-Identifier: CC-BY-NC-SA-4.0
%
% preamble.tex -- document configuration
% Copyright (C) 2022--2023  Jacob Koziej <jacobkoziej@gmail.com>

\usepackage{geometry}
\geometry{
	paper=b5paper,
}

\usepackage{amssymb}
\usepackage{csquotes}
\usepackage{graphicx}
\usepackage{luapackageloader}
\usepackage{ragged2e}

\usepackage{minted}
\setminted{
	breaklines,
	linenos,
	obeytabs,
}

\usepackage{hyperref}
\hypersetup{
	colorlinks,
	pdfinfo = {
		Title    = A Zoomer's Guide to C Programming,
		Author   = Jacob Koziej,
		Subject  = C Programming,
		Keywords = {C, programming},
	},
}
\usepackage{bookmark}

\setcounter{chapter}{-1}

\newcounter{exercise}
\counterwithin{exercise}{chapter}

\newcounter{solution}
\counterwithin{solution}{section}

\directlua{zgtc = require('zgtc')}

\NewDocumentCommand{\commandOutput}{m}{
	\inputminted[
		fontsize=\small,
		linenos=false,
		style=ansi-color,
	]{ansi-color}{#1}
}

\NewDocumentCommand{\inputlicenselessminted}{mm}{
	\inputminted[
		fontsize=\small,
		lastline=\directlua{tex.print(zgtc.lines('#2'))},
	]{#1}{#2}
}

\NewDocumentCommand{\inputlinesminted}{mmmm}{
	\inputminted[
		firstline=#3,
		lastline=#4,
	]{#1}{#2}
}

\NewDocumentCommand{\exercise}{m}{%
	\refstepcounter{exercise}%
	\par\noindent%
	\(\spadesuit\)~\textsl{\texttt{Exercise~\theexercise}}%
	\par\noindent\unskip\ignorespaces%
	{\input{#1}}%
	\par\ignorespacesafterend\vspace{0.2em}%
}

\NewDocumentCommand{\footurl}{m}{\footnote{\url{#1}}}

\NewDocumentCommand{\key}{m}{\texttt{#1}}

\NewDocumentCommand{\solution}{m}{%
	\refstepcounter{solution}%
	\par\noindent%
	\(\nabla\)~\textsl{\texttt{Solution~\thesolution}}%
	\par\noindent\unskip\ignorespaces%
	{\input{#1}}%
	\par\ignorespacesafterend\vspace{0.2em}%
}

\NewDocumentCommand{\vocab}{m}{\textbf{\textsl{#1}}}

\makeatletter
\RenewDocumentCommand{\cleardoublepage}{}{
	\clearpage
	\if@twoside
		\ifodd\c@page\else
			\begingroup
			\hbox{}
			\thispagestyle{empty}
			\newpage
			\if@twocolumn
				\hbox{}
				\newpage
			\fi
			\endgroup
		\fi
	\fi
}
\makeatother

\input{version}
