% SPDX-License-Identifier: CC-BY-NC-SA-4.0
%
% hello-world.tex -- the timeless classic
% Copyright (C) 2022  Jacob Koziej <jacobkoziej@gmail.com>

\chapter{Hello World!}

Let's write a timeless classic in C.  Fire up your editor of choice and
type the following into \texttt{hello-world.c}:

\inputlicenselessminted{c}{hello-world.d/hello-world.c}

\noindent
Let's compile the program:
\commandOutput{hello-world.d/compile-hello-world.output}
\noindent
And let's run it:
\commandOutput{hello-world.d/run-hello-world.output}
\noindent
Awesome!  We just wrote our first C program!  That’s great, but what the
hell does any of this mean?

\section{What the Hell Does Any of This Mean?}

There's no point in beating around the bush; the only way we're going to
understand any of this is by breaking it down, line by line!

\inputlinesminted{c}{hello-world.d/hello-world.c}{1}{1}

\noindent
Our first line concerns us with a \vocab{C preprocessor directive}.
These directives are picked up by the \vocab{preprocessor} and modify
the contents of our source code before handing it off to the compiler.
In our case, we're asking for the preprocessor to include the contents
of the \mintinline{c}{stdio.h} \vocab{header} in our source code.  We'll
cover where the preprocessor finds these files in a later section when
we cover the compilation process, but for now, all you need to know is
that by including a file, we gain access to library functions.

\inputlinesminted{c}{hello-world.d/hello-world.c}{4}{4}

\noindent
Here we have the beginning of a \vocab{function definition}.  Simmilar
to how we may define a word with a sequence of other words, we can
define a function as a sequence of operations and other functions.
Unlike the definition of words, the compiler needs some help
understanding where our new function fits into the world; this comes in
the form of an optional \vocab{return type} and an optional list of
\vocab{parameters}. In our case, we intend to return an \mintinline%
{c}{int} and denote no parameters with \mintinline{c}{void}.  What's
also important to note here is that \mintinline{c}{main()} is a special
function as it serves as our program's \vocab{entry point}.

\inputlinesminted{c}{hello-world.d/hello-world.c}{5}{5}

\noindent
Although we've told the compiler our intentions for a function
definition, we must now somehow delineate the start of the
\vocab{function body}.  In C, we do this with a left curly brace.

\inputlinesminted{c}{hello-world.d/hello-world.c}{6}{6}

\noindent
Here is our first \vocab{function call}.  In C, we call functions by
writing their \vocab{identifier} and appending a set of parenthesis with
the required \vocab{arguments}.  Although not shown here, we can also
store the returned value of non-\mintinline{c}{void} function calls.

For our call to \mintinline{c}{printf()}, we've provided the \vocab{%
string literal} argument, \mintinline{c}{"Hello World!\n"}.  What's
important to notice here is the \mintinline{c}{'\n'} at the end of the
string.  We call this an \vocab{escape sequence}, as it allows us to
easily type special characters such as newline.

\inputlinesminted{c}{hello-world.d/hello-world.c}{8}{8}

\noindent
Before we finish our function, we must specify a return value.  What's
special about \mintinline{c}{main()} is that the value we return here is
interpreted as our program's \vocab{exit code} by the operating system.
By convention, a value of zero is considered a sucessful execution of a
program.  Conversely, any non-zero exit code is an error.  Often times,
programs return different non-zero values here to signify different
types of unrecoverable errors.

\inputlinesminted{c}{hello-world.d/hello-world.c}{9}{9}

\noindent
Now that we're all done, we need to close the body of our function with
a right curly brace.

\subsection{Safe at Last!?}

I don't know about you, but I don't think that was half bad. \emph{%
Sure}, I may have thrown a lot of new terms at you, and you may have
more questions than when we started, but look at the bright side: you're
soon going to forget that you don't understand.

Unfortunately for us, we're dealing with a circular dependency, and very
little is going to make sense until we get through at least a complete
cycle.  Buckle up and get ready to be overwhelmed---we're about to go off
the deep end!

\section{Out the Frying Pan and into the Fire}

To better understand what we're dealing with here, I think it makes
sense for us to extend our Hello World program.  Reopen \texttt{%
hello-world.c} and modify it to look like so:
\inputlicenselessminted{c}{hello-world.d/compile-error.c}
\noindent
Let's go and recompile it to see what happens:
\commandOutput{hello-world.d/compile-error.output}
\noindent
Well, that's a bummer.  What went wrong?

\subsection{Compilation Errors}

What we just ran into is what's called a \vocab{compilation error}.
This kind of error occurs when we have a \vocab{syntax error} in our
source code.  The compiler is programmed to follow a grammar that
describes the C language, nothing more, nothing less.  That is not to
say compiler writers haven't gone the extra mile to make our lives a
little easier by adding descriptive error messages.

Note how the compiler output is formatted.  Each error is prefaced by a
file path, next by a line number, and finally, the position in the
offending line.  Oh and look, the compiler writers were even nice enough
to print a big arrow as to where they think we should add our missing
semicolon.

\

\noindent
Let's get that fixed and see if we're in the clear:
\inputlicenselessminted{c}{hello-world.d/compile-warning.c}
\noindent
\commandOutput{hello-world.d/compile-warning.output}
\noindent
\commandOutput{hello-world.d/run-compile-warning.output}
\noindent

\

\noindent
So close and yet so far!

\subsection{Compilation Warnings}

Even though our program compiled and ran, I wouldn't say this is the
output we expected!  This time around, we've encountered a \vocab{%
compilation warning}.  Unlike a compilation error, it isn't fatal to the
compilation process.  This means that the compiler can begrudgingly
translate our code into machine instructions that might not entirely
make sense to execute!  Like a hot stove, just because we can touch it
doesn't necessarily mean we should, at least not yet \texttt{;)}

Generally speaking, you want to avoid having any compilation warnings.
There are certain situations where the compiler \textbf{\emph{may}} be
wrong, with a huge emphasis on the \emph{may}, as this is rarely the
case.  \textbf{IF} you do come across a compilation warning in the field
which you know can be safely ignored, you can disable said warning by
supplying the compiler flag \texttt{-Wno-[warning-name]}.

\

\noindent
Let's do that for our current code to see it in action:
\commandOutput{hello-world.d/compile-warning-disable.output}
\noindent
Notice how there's no output, and yet there's a problem!  Use this
knowledge sparingly and with great care; there's no knowing how many
hours of your life this could cost you!

\

\noindent
Pivoting from what \emph{not} to do, how about we start enabling all the
warnings?  Counter-intuitively, \texttt{-Wall} does \underline{not}
enable all warnings.  For that, you'll need to toggle the \texttt{%
-Wextra} warnings, and because compiler writers can be insufferable
bastards, you also need to also enable the \texttt{-Wpedantic} warnings.
