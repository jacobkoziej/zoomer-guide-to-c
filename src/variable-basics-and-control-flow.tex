% SPDX-License-Identifier: CC-BY-NC-SA-4.0
%
% variable-basics-and-control-flow.tex -- lets do something
% Copyright (C) 2022  Jacob Koziej <jacobkoziej@gmail.com>

\chapter{Variable~Basics \&~Control~Flow}

So far, we haven't gotten to write many meaningful programs because of
our limited tool set.  Here is where \vocab{variables} and \vocab{%
control flow} come in to save the day, and well, make programming,
programming.

\section{Variables}

If you've ever had the great fortune of solving an algebraic expression,
you've already interacted with variables, at least in the domain of
mathematics.  Much how we may use variables in mathematics to describe
behavior for various values, variables in programming allow us to write
more generalized code, allowing us to masquerade \vocab{constants} as
identifiers.

\

\noindent
So far, we've only seen the following example of a variable:

\begin{minted}[fontsize=\small, linenos=false]{c}
const int epoch = 1970;
\end{minted}

\noindent
Even though it is single line of C, there is quite a lot to unpack here.
