% SPDX-License-Identifier: CC-BY-NC-SA-4.0
%
% variable-basics-and-control-flow.tex -- lets do something
% Copyright (C) 2022  Jacob Koziej <jacobkoziej@gmail.com>

\chapter{Variable~Basics \&~Control~Flow}

So far, we haven't gotten to write many meaningful programs because of
our limited tool set.  Here is where \vocab{variables} and \vocab{%
control flow} come in to save the day, and well, make programming,
programming.

\section{Variables}

If you've ever had the great fortune of solving an algebraic expression,
you've already interacted with variables, at least in the domain of
mathematics.  Much how we may use variables in mathematics to describe
behavior for various values, variables in programming allow us to write
more generalized code, allowing us to masquerade \vocab{constants} as
identifiers.

\

\noindent
So far, we've only seen the following example of a variable:

\begin{minted}[fontsize=\small, linenos=false]{c}
const int epoch = 1970;
\end{minted}

\noindent
Even though it is single line of C, there is quite a lot to unpack here.

\subsection{Primitive Data Types}

C is what's considered a \vocab{strictly typed} language meaning every
variable \emph{must} have a \vocab{data type} associated with it.
Luckily, C only has a handful of \vocab{primitive data types}, which
Table~\ref{table:variable-basics-and-control-flow:c-primitive-data-types}
neatly showcases.

\providecommand{\zgtcchar}{
Much as the name might suggest, a \mintinline{c}{char} stores character
types.  What makes \mintinline{c}{char} special is that it will always
be the smallest integer type on a \vocab{host architecture} but always
at least one byte in size.
}
\providecommand{\zgtcint}{
An \mintinline{c}{int} is the default size integer for a platform.
Typically this is the \vocab{word size} of the processor.  By default,
integer constants are of type \mintinline{c}{int}.
}
\providecommand{\zgtclong}{
When an \mintinline{c}{int} isn't big enough to store an integer
constant, we can opt to use a \mintinline{c}{long}.  Typically, these
are double the size of an \mintinline{c}{int}.  If we'd like to write a
\mintinline{c}{long} as an integer constant, we need to append either
\mintinline{c}{l} or \mintinline{c}{L}.
}
\providecommand{\zgtcfloat}{
We can use a \mintinline{c}{float} to represent real numbers, or as
programmers tend to call them, floating-point constants.  Unlike
integers, they are encoded as exponentials in memory, meaning we can't
get arbitrary precision.  To write a floating-point constant, we have to
append either \mintinline{c}{f} or \mintinline{c}{F}.
}
\providecommand{\zgtcdouble}{
As the name may suggest, a \mintinline{c}{double} is to a
\mintinline{c}{float} as a \mintinline{c}{long} is to an
\mintinline{c}{int}.  By default, floating-point constants are of type
\mintinline{c}{double}.
}

\begin{table}[h!]
\centering
\begin{tabular}{lp{0.6\textwidth}l}
\toprule
Data Type & Description & Example \\
\midrule
\mintinline{c}{char}   & \zgtcchar   & \mintinline{c}{'c'}   \\
\mintinline{c}{int}    & \zgtcint    & \mintinline{c}{3}     \\
\mintinline{c}{long}   & \zgtclong   & \mintinline{c}{3l}    \\
\mintinline{c}{float}  & \zgtcfloat  & \mintinline{c}{3.14f} \\
\mintinline{c}{double} & \zgtcdouble & \mintinline{c}{3.14}  \\
\bottomrule
\end{tabular}
\caption{C Primitive Data Types}
\label{table:variable-basics-and-control-flow:c-primitive-data-types}
\end{table}

\noindent
Unfortunately for us, this is a gross simplification of C's primitive
data types, but we'll see in later chapters that strings and
floating-point types are surprisingly complicated topics.

If it wasn't already evident, programming concepts rely on thousands of
little lies, or as programmers like to put it, to feel better about
themselves, these concepts have been \emph{\enquote{abstracted away.}}
As hard as it might be to believe, thousands of small lies can allow us
to build quite a strong foundation.  It's only after we've built up this
strong foundation that it becomes our responsibility to go back and
reinforce it with the truth, or at the very least, fewer lies.
