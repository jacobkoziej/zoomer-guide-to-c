% SPDX-License-Identifier: CC-BY-NC-SA-4.0
%
% preface.tex -- subject, scope, and aims
% Copyright (C) 2022  Jacob Koziej <jacobkoziej@gmail.com>

\chapter{Preface}

\inputlicenselessminted{c}{preface.d/Q_rsqrt.c}

\noindent
Fascinating, right?  What we're looking at here is Quake III's fast
inverse square root\footnote{\url{https://github.com/id-Software/%
Quake-III-Arena/blob/master/code/game/q_math.c\#L552}}, and it's
absolutely \emph{genius}.  When I think of C programming, this is the
first thing that comes to mind: pure cleverness and a deep understanding
of hardware features.  Of course, we still have a long way ahead of us
before you can truly appreciate the beauty of this code, but that's
alright.  After all, isn't that why you're here?

If it wasn’t clear enough yet, C is rather \emph{bizarre}.  Not only
does it let you do almost anything, but there’s little to nothing
stopping you from shooting yourself in the foot.  Even the code I showed
just now contains undefined behavior!

That said, learning C in \directlua{tex.print(tex.year)} is still a
worthwhile investment.  Admittedly, I am biased as I adore C in all its
ugliness, but the truth of the matter still stands: learning C will
\emph{force} you to become a better programmer.  There’s just something
about writing complex software with the programming equivalent of stone
tools that necessitates a good grasp of code writing.

Although there are a lot of great free resources out there the learn C,
I feel like they all fall short in some way.  There is either little to
no guidance for future topics, too little or too many assumptions about
background knowledge, or little to no exercises to reinforce freshly
covered content.  You can absolutely learn C by browsing these
resources, as I can attest to that, but I feel that it requires a
significant amount of patience and a bit of luck to ensure you find your
way down the right path.  Not to mention the extra effort of finding
effective ways of reinforcing what you had just learned.

I won’t be making too many assumptions about your background knowledge,
but a background in some programming principles will help; even
something as simple as Berkeley’s Snap!\footnote{\url{https://%
snap.berkeley.edu/}} or MIT’s Scratch\footnote{\url{https://%
scratch.mit.edu/}} is more than enough.  With that out of the way, how
about we get right to it?
