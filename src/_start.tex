% SPDX-License-Identifier: CC-BY-NC-SA-4.0
%
% start.tex -- first instructions
% Copyright (C) 2022  Jacob Koziej <jacobkoziej@gmail.com>

\chapter{\texttt{\_start}}

It only feels right to \mintinline{c}{_start} with a few instructions to
get us ready for the \mintinline{c}{main()} matter of the book.  Jokes
aside, there are a few things that are important to touch upon before we
cover any material.  Much how \mintinline{c}{_start} prepares our
program before its call to \mintinline{c}{main()}, we must do the same
as the execution of the rest of this book depends on us soundly getting
off the ground!

\section{A Unix-Like Environment}

This book will be running under the assumption that you're working in a
Unix-like environment, more specifically, a Linux-based environment.
macOS users, you'll most likely be able to follow along for most of the
book, but beware, there are slight nuances between Linux and macOS!
Windows users, sorry, but you're out of luck here, though you may have
varied luck with the Windows Subsystem for Linux\footnote{\url{https://%
learn.microsoft.com/en-us/windows/wsl/about}}.

If you are not already working in a Linux-based environment, I highly
recommend you get into one while going through this book, as it's the
most convenient to work with for C development.  Possible options
include, but are not limited to: a full Linux install, a dual-boot Linux
install, or a virtual machine installation (though the latter is rather
lackluster).

\section{The Right Editor}

Choosing the right editor is imperative if you are to benefit the most
from this book.  It is for this very reason I \emph{strongly} advise
against using \textbf{any} form of IDE.  The reason for this is simple:
they hide the steps required to get from source code to an executable
that your machine can run.  An IDE is fine for a seasoned C programmer;
but detrimental to someone just starting, as you'll have no idea how a
compilation error came to fruition. If you're struggling to find a good
editor, I recommend Neovim\footnote{\url{https://neovim.io/}}.

Another thing is autocompletion: it is the bane of all evil when you're
learning a language.  It's understandable if you wish to be fast, but
you're here to \emph{learn}, not press \key{Tab}!  Disable it if your
editor enables it by default.  You'll thank me later.

\section{Indentation}

Setting your editor's indents to 8 characters wide does two things: it
makes indents clear and makes it irritating to write deeply nested code.
The latter point is of particular importance.

Many editors allow you to add a colored column at a specified position.
I recommend adding one at 80 characters.  The number 80 originates from
when terminals were \(24 \times 80\) characters in size.  Once added,
don't allow your code to exceed this limit (within reason).

It may seem strange to intentionally constrain ourselves to what used to
be a technological limitation, but it serves a great purpose: it forces
us to write better code!  Once indentation exceeds a few levels, you'll
notice that it becomes exhausting to stay within the 80-character limit;
this is a telltale sign that it may be time to step back and restructure
your code.  Do this, and you'll avoid what I call nesting hell!

\section{File Organization \& Version Control}

Keeping a well-organized file structure is critical to building complex
software.  This book follows a systematic format for the exercises in
each chapter, so it shouldn't be too difficult to create a file
structure that best works for \emph{you}.  I'm intentionally keeping
this vague as I believe learning how to best structure projects comes
from trial and error, browsing other people's projects, and taking notes
on what does and doesn't work.

I would also like to recommend creating a Git\footnote{\url{https://%
git-scm.com/}} repository to track your work for the exercises.  A
version control system is important for maintaining any complex project,
as it allows for safe experimentation and easy rollbacks, among many
other great things!  Again, I'm intentionally keeping this vague to
encourage you to explore Git on your own time.  If you're new to Git or
have never heard of it before, I highly recommend starting with MIT's
Missing Semester lecture\footnote{\url{https://missing.csail.mit.edu/%
2020/version-control/}}.

\section{Breaks}

I think it goes without saying, but you're \emph{not} going to become a
C programmer overnight.  There's no shortcut I know of around all the
head-banging and hair-pulling.  All I can do is provide you with the
best path I can pave; it's up to you to do the walking.

You'll learn more if you learn in lots of small chunks than a few big
blocks of your time.  That time between sessions is when your brain has
the time to digest all the information it has just absorbed.  Don't rob
your mind of that thinking time.  You'll be amazed by all the places
that'll take you.

Most of all, don't let frustration get the best of you.  An important
skill to have is knowing when to stop.  Take a break, and come back with
a fresh mind.  I guarantee you'll have a different perspective on what's
blocking you.

\section{Solutions}

As my final warning, resist the urge to check solutions to exercises
until you've exhausted all other resources.  You're robbing yourself of
a learning experience otherwise.  Solutions are a double-edged sword:
use them to get yourself out of a dead end, not as a crutch to cheat
your way through this book!
